% Options for packages loaded elsewhere
\PassOptionsToPackage{unicode}{hyperref}
\PassOptionsToPackage{hyphens}{url}
%
\documentclass[
]{article}
\usepackage{amsmath,amssymb}
\usepackage{lmodern}
\usepackage{iftex}
\ifPDFTeX
  \usepackage[T1]{fontenc}
  \usepackage[utf8]{inputenc}
  \usepackage{textcomp} % provide euro and other symbols
\else % if luatex or xetex
  \usepackage{unicode-math}
  \defaultfontfeatures{Scale=MatchLowercase}
  \defaultfontfeatures[\rmfamily]{Ligatures=TeX,Scale=1}
\fi
% Use upquote if available, for straight quotes in verbatim environments
\IfFileExists{upquote.sty}{\usepackage{upquote}}{}
\IfFileExists{microtype.sty}{% use microtype if available
  \usepackage[]{microtype}
  \UseMicrotypeSet[protrusion]{basicmath} % disable protrusion for tt fonts
}{}
\makeatletter
\@ifundefined{KOMAClassName}{% if non-KOMA class
  \IfFileExists{parskip.sty}{%
    \usepackage{parskip}
  }{% else
    \setlength{\parindent}{0pt}
    \setlength{\parskip}{6pt plus 2pt minus 1pt}}
}{% if KOMA class
  \KOMAoptions{parskip=half}}
\makeatother
\usepackage{xcolor}
\usepackage[margin=1in]{geometry}
\usepackage{color}
\usepackage{fancyvrb}
\newcommand{\VerbBar}{|}
\newcommand{\VERB}{\Verb[commandchars=\\\{\}]}
\DefineVerbatimEnvironment{Highlighting}{Verbatim}{commandchars=\\\{\}}
% Add ',fontsize=\small' for more characters per line
\usepackage{framed}
\definecolor{shadecolor}{RGB}{248,248,248}
\newenvironment{Shaded}{\begin{snugshade}}{\end{snugshade}}
\newcommand{\AlertTok}[1]{\textcolor[rgb]{0.94,0.16,0.16}{#1}}
\newcommand{\AnnotationTok}[1]{\textcolor[rgb]{0.56,0.35,0.01}{\textbf{\textit{#1}}}}
\newcommand{\AttributeTok}[1]{\textcolor[rgb]{0.77,0.63,0.00}{#1}}
\newcommand{\BaseNTok}[1]{\textcolor[rgb]{0.00,0.00,0.81}{#1}}
\newcommand{\BuiltInTok}[1]{#1}
\newcommand{\CharTok}[1]{\textcolor[rgb]{0.31,0.60,0.02}{#1}}
\newcommand{\CommentTok}[1]{\textcolor[rgb]{0.56,0.35,0.01}{\textit{#1}}}
\newcommand{\CommentVarTok}[1]{\textcolor[rgb]{0.56,0.35,0.01}{\textbf{\textit{#1}}}}
\newcommand{\ConstantTok}[1]{\textcolor[rgb]{0.00,0.00,0.00}{#1}}
\newcommand{\ControlFlowTok}[1]{\textcolor[rgb]{0.13,0.29,0.53}{\textbf{#1}}}
\newcommand{\DataTypeTok}[1]{\textcolor[rgb]{0.13,0.29,0.53}{#1}}
\newcommand{\DecValTok}[1]{\textcolor[rgb]{0.00,0.00,0.81}{#1}}
\newcommand{\DocumentationTok}[1]{\textcolor[rgb]{0.56,0.35,0.01}{\textbf{\textit{#1}}}}
\newcommand{\ErrorTok}[1]{\textcolor[rgb]{0.64,0.00,0.00}{\textbf{#1}}}
\newcommand{\ExtensionTok}[1]{#1}
\newcommand{\FloatTok}[1]{\textcolor[rgb]{0.00,0.00,0.81}{#1}}
\newcommand{\FunctionTok}[1]{\textcolor[rgb]{0.00,0.00,0.00}{#1}}
\newcommand{\ImportTok}[1]{#1}
\newcommand{\InformationTok}[1]{\textcolor[rgb]{0.56,0.35,0.01}{\textbf{\textit{#1}}}}
\newcommand{\KeywordTok}[1]{\textcolor[rgb]{0.13,0.29,0.53}{\textbf{#1}}}
\newcommand{\NormalTok}[1]{#1}
\newcommand{\OperatorTok}[1]{\textcolor[rgb]{0.81,0.36,0.00}{\textbf{#1}}}
\newcommand{\OtherTok}[1]{\textcolor[rgb]{0.56,0.35,0.01}{#1}}
\newcommand{\PreprocessorTok}[1]{\textcolor[rgb]{0.56,0.35,0.01}{\textit{#1}}}
\newcommand{\RegionMarkerTok}[1]{#1}
\newcommand{\SpecialCharTok}[1]{\textcolor[rgb]{0.00,0.00,0.00}{#1}}
\newcommand{\SpecialStringTok}[1]{\textcolor[rgb]{0.31,0.60,0.02}{#1}}
\newcommand{\StringTok}[1]{\textcolor[rgb]{0.31,0.60,0.02}{#1}}
\newcommand{\VariableTok}[1]{\textcolor[rgb]{0.00,0.00,0.00}{#1}}
\newcommand{\VerbatimStringTok}[1]{\textcolor[rgb]{0.31,0.60,0.02}{#1}}
\newcommand{\WarningTok}[1]{\textcolor[rgb]{0.56,0.35,0.01}{\textbf{\textit{#1}}}}
\usepackage{graphicx}
\makeatletter
\def\maxwidth{\ifdim\Gin@nat@width>\linewidth\linewidth\else\Gin@nat@width\fi}
\def\maxheight{\ifdim\Gin@nat@height>\textheight\textheight\else\Gin@nat@height\fi}
\makeatother
% Scale images if necessary, so that they will not overflow the page
% margins by default, and it is still possible to overwrite the defaults
% using explicit options in \includegraphics[width, height, ...]{}
\setkeys{Gin}{width=\maxwidth,height=\maxheight,keepaspectratio}
% Set default figure placement to htbp
\makeatletter
\def\fps@figure{htbp}
\makeatother
\setlength{\emergencystretch}{3em} % prevent overfull lines
\providecommand{\tightlist}{%
  \setlength{\itemsep}{0pt}\setlength{\parskip}{0pt}}
\setcounter{secnumdepth}{5}
\ifLuaTeX
  \usepackage{selnolig}  % disable illegal ligatures
\fi
\IfFileExists{bookmark.sty}{\usepackage{bookmark}}{\usepackage{hyperref}}
\IfFileExists{xurl.sty}{\usepackage{xurl}}{} % add URL line breaks if available
\urlstyle{same} % disable monospaced font for URLs
\hypersetup{
  pdftitle={Trajectories},
  pdfauthor={Kadri Künnapuu; Kadri Ligi; Raivo Kolde; Sven Laur; Solomon Ioannou; Peter Rijnbeek; Jaak Vilo; Sulev Reisberg},
  hidelinks,
  pdfcreator={LaTeX via pandoc}}

\title{Trajectories}
\author{Kadri Künnapuu \and Kadri Ligi \and Raivo Kolde \and Sven
Laur \and Solomon Ioannou \and Peter Rijnbeek \and Jaak Vilo \and Sulev
Reisberg}
\date{}

\begin{document}
\maketitle

{
\setcounter{tocdepth}{3}
\tableofcontents
}
\hypertarget{introduction}{%
\section{Introduction}\label{introduction}}

This vignette describes how to use the Trajectories package in discovery
and validation mode. The discovery mode enables to discover and
visualize event pairs in OMOP-formatted observational health data while
the validation mode validates the results against event pairs discovered
elsewhere.

We have selected type 2 diabetes as an example, creating graphs with
both conditions and drug eras included in the analysis.

The package can be run in 2 modes: discovery and validation. Both modes
have separate CodeToRun\ldots R files in the \texttt{extras} folder.

\hypertarget{installation-instructions}{%
\section{Installation instructions}\label{installation-instructions}}

\begin{enumerate}
\def\labelenumi{\arabic{enumi}.}
\tightlist
\item
  Clone the source code from \url{https://github.com/EHDEN/Trajectories}
  repository
\item
  Open the project (file \textbf{\emph{Trajectories.Rproj}}) in RStudio
\item
  Install the package via top-right menu: \textbf{\emph{Build}}
  -\textgreater{} \textbf{\emph{Install and Restart}}
\end{enumerate}

\hypertarget{running-the-package}{%
\section{Running the package}\label{running-the-package}}

In order to run the package, you need:

\begin{enumerate}
\def\labelenumi{\arabic{enumi}.}
\tightlist
\item
  A database that has data in OMOP CDM v5 format. The database should
  also contain OMOP vocabulary, but this can be in a separate schema.
\item
  A database user + passwords that has:

  \begin{itemize}
  \tightlist
  \item
    Read (SELECT) permission from OMOP CDM tables and vocabulary
  \item
    CREATE, DROP, SELECT, INSERT, UPDATE, DELETE permission in some
    schema of the same database. This is used for creating and temporary
    analysis tables. For this, you can create a separate schema in the
    same database.
  \end{itemize}
\end{enumerate}

In case you do not have an access to any OMOP CDM, you can use test data
from Eunomia package. To run a package on that data, simply run all
commands from \texttt{./extras/CodeToRunEunomia.R}

\hypertarget{setting-up-the-database-login-credentials}{%
\subsection{Setting up the database login
credentials}\label{setting-up-the-database-login-credentials}}

As the package needs to connect to the database, you have to add login
credentials into \textbf{\emph{.Renviron}} file:

\begin{enumerate}
\def\labelenumi{\arabic{enumi}.}
\tightlist
\item
  Rename \textbf{\emph{Renviron.template}} to \textbf{\emph{.Renviron}}
\item
  Add your database username and password to \textbf{\emph{.Renviron}}
  (these are the database credentials for accessing OMOP CDM and writing
  temporary analysis tables/data). After you restart your RStudio, it
  automatically reads the database credentials from that file so that
  you do not have to enter them each time. Also, .Renviron is not under
  version control, therefore it is kept unchanged even if you pull the
  updates of the R-package.
\item
  Restart RStudio to automatically read in database credentials.
\end{enumerate}

\hypertarget{defining-the-study-cohort-and-analysis-settings}{%
\subsection{Defining the study cohort and analysis
settings}\label{defining-the-study-cohort-and-analysis-settings}}

The package searches for event trajectories within cohort. This means
that you either \textbf{\emph{have to define a cohort by yourself}} or
\textbf{\emph{use some built-in cohort}} (or a cohort defined by someone
else).

The built-in cohorts + analysis settings are located in
\texttt{./inst/extdata/} folder. For example, the validation of the
event pairs of Type 2 diabetes cohort can be found from
\texttt{./inst/extdata/T2D-validate}. If you are going to use built-in
setup, skip the rest of this section

If you need to create a new cohort + analysis, follow these steps:

\begin{enumerate}
\def\labelenumi{\arabic{enumi}.}
\tightlist
\item
  Create an empty folder
\item
  Define the cohort in your Atlas server
  (\url{https://atlas.ohdsi.org/})
\item
  Export SQL of the cohort definition (format ``SQL Server'') to
  \texttt{cohort.sql} file and save it to your created folder.
\item
  Take \texttt{./inst/extdata/T2D/trajectoryAnalysisArgs.json} file and
  copy it to the same folder.
\item
  Modify the parameters (analysis settings) in
  \texttt{trajectoryAnalysisArgs.json} to fit your needs.
\end{enumerate}

For more information of the parameters, run

\texttt{?Trajectories::createTrajectoryAnalysisArgs}

\hypertarget{setting-up-study-parameters-and-running-the-package}{%
\subsection{Setting up study parameters and running the
package}\label{setting-up-study-parameters-and-running-the-package}}

The easiest way to run the package is to open either
\texttt{./extras/CodeToRunDiscover.R} or
\texttt{./extras/CodeToRunValidate.R} depending on whether you wish to
find new event trajectories or you are just validating the pairs from
someone's result.

In order to validate the event pairs of Type 2 Diabetes that were
reported by Kunnapuu et al., open \texttt{./extras/CodeToRunValidate.R}.

You have to edit some lines in the file before running it. As both files
are quite similar, so the guidelines are common.

First, add correct connection string to

\begin{Shaded}
\begin{Highlighting}[]
\NormalTok{connectionString }\OtherTok{=} \StringTok{"jdbc:postgresql://[host]:[port]/[database]"}
\end{Highlighting}
\end{Shaded}

Second, set up the correct values when calling
\texttt{createTrajectoryLocalArgs()}

\begin{Shaded}
\begin{Highlighting}[]
\NormalTok{oracleTempSchema }\OtherTok{=} \StringTok{"temp\_schema"}\NormalTok{, }\CommentTok{\#no need to change this even when you are not using Oracle}
\NormalTok{prefixForResultTableNames }\OtherTok{=} \StringTok{""}\NormalTok{, }\CommentTok{\#It is a prefix that is used for creating temporary table names.}
             \CommentTok{\#You can be set to "".}
\NormalTok{cdmDatabaseSchema }\OtherTok{=} \StringTok{\textquotesingle{}...\textquotesingle{}}\NormalTok{, }\CommentTok{\#It is the name of database schema where the OMOP CDM data is actually}
             \CommentTok{\#kept/taken}
\NormalTok{vocabDatabaseSchema }\OtherTok{=} \StringTok{\textquotesingle{}...\textquotesingle{}}\NormalTok{, }\CommentTok{\#It is the name of database schema where the OMOP vocabularies are}
             \CommentTok{\#kept/taken}
\NormalTok{resultsSchema }\OtherTok{=} \StringTok{\textquotesingle{}...\textquotesingle{}}\NormalTok{, }\CommentTok{\#It is the name of database schema where the analysis tables will be}
             \CommentTok{\#created}
\NormalTok{sqlRole }\OtherTok{=}\NormalTok{ F, }\CommentTok{\#You may always use \textquotesingle{}F\textquotesingle{}. Setting specific role might be useful in PostgreSQL when}
             \CommentTok{\#you want to create tables by using specific role so that the others also see the}
             \CommentTok{\#results. However, then you must ensure that this role has permissions to read from}
             \CommentTok{\#all necessary schemas and also create tables to resultsSchema}
\NormalTok{inputFolder}\OtherTok{=}\FunctionTok{system.file}\NormalTok{(}\StringTok{"extdata"}\NormalTok{, }\StringTok{"T2D{-}validate"}\NormalTok{, }\AttributeTok{package =} \StringTok{"Trajectories"}\NormalTok{), }\CommentTok{\# Full path to}
             \CommentTok{\#input folder that contains SQL file for cohort definition and}
             \CommentTok{\#trajectoryAnalysisArgs.json. You can use built{-}in folders of this package such as:}
             \CommentTok{\#inputFolder=system.file("extdata", "T2D{-}validate", package = "Trajectories").}
             \CommentTok{\#Otherwise use full path: inputFolder=\textquotesingle{}/here/is/my/path\textquotesingle{}}
\NormalTok{mainOutputFolder}\OtherTok{=}\StringTok{\textquotesingle{}/here/is/my/path\textquotesingle{}}\NormalTok{, }\CommentTok{\#Path to general folder where all the outputs of the package}
             \CommentTok{\#will be written. The folder must exist. Each database and analysis will have a}
             \CommentTok{\#separate subfolder automatically.}
\NormalTok{databaseHumanReadableName}\OtherTok{=}\StringTok{\textquotesingle{}...\textquotesingle{}} \CommentTok{\#Use something short. This will be used as a folder name an it will}
             \CommentTok{\#be added to the titles of the graph.}
\end{Highlighting}
\end{Shaded}

For more information about the parameters you can always use

\begin{Shaded}
\begin{Highlighting}[]
\NormalTok{?createTrajectoryLocalArgs}
\end{Highlighting}
\end{Shaded}

In order to validate the event pairs of Type 2 Diabetes that were
reported by Kunnapuu et al., use

\begin{Shaded}
\begin{Highlighting}[]
\NormalTok{inputFolder}\OtherTok{=}\FunctionTok{system.file}\NormalTok{(}\StringTok{"extdata"}\NormalTok{, }\StringTok{"T2D{-}validate"}\NormalTok{, }\AttributeTok{package =} \StringTok{"Trajectories"}\NormalTok{)}
\end{Highlighting}
\end{Shaded}

Now you have all set and you can simply run the workhorse
\texttt{Trajectories::discover()} or \texttt{Trajectories::validate()}.

Both processes are made of several subprocesses which start one after
another. For your convenience, you can change TRUE/FALSE flags when
calling these subprocesses. Run

\begin{Shaded}
\begin{Highlighting}[]
\NormalTok{?Trajectories}\SpecialCharTok{::}\NormalTok{discover}
\end{Highlighting}
\end{Shaded}

or

\begin{Shaded}
\begin{Highlighting}[]
\NormalTok{?Trajectories}\SpecialCharTok{::}\NormalTok{validate}
\end{Highlighting}
\end{Shaded}

for more information!

\hypertarget{assessing-the-results}{%
\subsection{Assessing the results}\label{assessing-the-results}}

All results are created to the folder that was specified by parameters
\textbf{\emph{mainOutputFolder}} and
\textbf{\emph{databaseHumanReadableName}}.

Note that subfolder \textbf{\emph{validation\_setup}} is always created.
This is the folder that you can pass to someone else to validate your
results in their database - it contains both the cohort definition and
analysis settings.

\end{document}
